\begin{resumo}
A robótica móvel preocupa-se em trazer máquinas capazes de locomover-se de forma autônoma. Para isso este objetivo foi dividido em desafios menores que precisavam ser tratados. Um deles e o abordado neste trabalho é a definição da trajetória do caminho a ser percorrido. Considerando que o ambiente em volta já é conhecido, o robô precisa definir um caminho seguro para chegar aonde deseja sem se chocar com os obstáculos existentes. Diversos algoritmos para isso foram desenvolvidos, sendo o objetivo deste trabalho fornecer um \textit{framework} que forneça tais soluções já prontas ao programador. Através de padrões de projetos o \textit{framework} é projetado para a fácil troca entre os algoritmos sem afetar o resto do código. Além da implementação destes algoritmos é projetado uma arquitetura que permita a portabilidade para o maior número de plataformas diferentes, preocupando-se tanto com reuso quanto com a legibilidade do código. O \textit{framework} se comunica com os outros módulos de movimentação do robô por uma interface simples e comum a todos os algoritmos. É importante ressaltar que o módulo de controle do hardware do robô e movimentá-lo de fato não faz parte do escopo deste trabalho, e sim conversar com este e fornecer a ele uma série de coordenadas para onde o robô deve ir. Este trabalho será \textit{open-source}, desenvolvido em linguagem Java e testado em robôs \textit{differential steering} do kit educacional da LEGO Mindstorm NXT.

 \vspace{\onelineskip}
    
 \noindent
 \textbf{Palavras-chaves}: robótica móvel. definição de trajetória. desenvolvimento de framework.
\end{resumo}
