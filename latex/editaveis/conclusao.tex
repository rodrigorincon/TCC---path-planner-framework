\chapter[Conclusão]{Conclusão}

Com este trabalho, espera-se facilitar o desenvolvimento de robôs para a comunidade de robótica educacional. Fornecendo o código sobre licença aberta, será dado a qualquer desenvolvedor o direito de implementar este \textit{framework} em sua aplicação e, com isso, diminuir pelo menos um pouco o retrabalho de construir um robô móvel.

Espera-se também concluir este trabalho com um \textit{framework} robusto, seguindo todas as boas práticas listadas na seção anterior e, principalmente, reutilizável. Será implementado os quatro algoritmos listados na seção 2.4 e a estrutura permitirá a inclusão de novos algoritmos. O crescimento do \textit{framework} com adição de novas funções pela comunidade é um objetivo esperado após o trabalho ser concluído e disponibilizado. Para tanto, será investido na modularidade do mesmo, buscando sempre uma arquitetura simples, pouco acoplada e coesa.

Espera-se também alcançar um alto nível de conhecimento no desenvolvimento de \textit{frameworks} e padrões de projeto. Um dos objetivos do trabalho é compreender essas práticas e o funcionamento dos robôs móveis e é esperado ao seu fim que estes conceitos sejam melhor compreendidos e trabalhados.