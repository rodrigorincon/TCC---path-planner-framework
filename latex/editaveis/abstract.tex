\begin{resumo}[Abstract]
 \begin{otherlanguage*}{english}
   The mobile robotics field is concerned in developing machines which move themselves autonomously. To achieve the movimentation problem was divided into smaller challenges that needed to be treated. One of these challenges, and the one addressed in this paper, is the path planning. Whereas the environment around is already known, the robot needs to define a secure path to get where it is headed without colliding with obstacles. Several algorithms for this were developed, and this paper aims to develop a framework that provides such ready-made solutions to the programmer. Through design patterns, the Traveller Framework is designed for easy switching between algorithms without affecting the rest of the code. Apart from the implementation of these algorithms, is designed an architecture was designed to allow portability to as many different platforms, worrying about reuse and code readability. The Traveller communicates with the other modules by a simple and common interface to all algorithms. Importantly, the control module of the hardware and move it in fact is not part of the scope of this paper, but chat with this module and gives it a series of coordinates for where it should go. This work will be open-source, developed in Java and tested on differential steering robots of the LEGO Mindstorm NXT educational robotics kit.

   \vspace{\onelineskip}
 
   \noindent 
   \textbf{Key-words}: mobile robotics. path planning. framework development. LEGO Mindstorm educational kit. Visibility Graph. Voronoi. Quadtree. Wavefront.
 \end{otherlanguage*}
\end{resumo}
