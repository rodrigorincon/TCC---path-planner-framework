\begin{resumo}[Abstract]
 \begin{otherlanguage*}{english}
   The mobile robotics field is concerned in developing machines which move themselves autonomously. The movimentation problem was divided into the challenges, understood as goals, lower. One of these challenges, the path planning, is adressed in this paper. Whereas  the environment around is already known, the robot needs to define a safe path to go where it is headed without colliding with any obstacles. Several algorithms for this were developed, such this paper aims to develop a framework that provides this ready-made solutions to the programmer. Through design patterns, the Traveller Framework is designed for easy switching between algorithms without affecting the rest of the code. Apart from the implementation of these algorithms, it is designed an architecture that allows portability to as many different platforms, worrying about reuse and code readability. The Traveller communicates with the other module robots by a simple and common interface to all algorithms. Importantly, the robot hardware control module and move is not part of the scope of this paper, but chat with this module and gives it a series of coordinates for where it should go. This work will be open-source, developed in Java and tested on differential steering robots of  the LEGO Mindstorm NXT educational robotics kit.

   \vspace{\onelineskip}
 
   \noindent 
   \textbf{Key-words}: mobile robotics. path planning. framework development. LEGO Mindstorm educational kit. Visibility Graph. Voronoi. Quadtree. Wavefront.
 \end{otherlanguage*}
\end{resumo}
