\begin{anexosenv}

\partanexos

\chapter{\textit{Host-spot Cards}}

Aqui será apresentado os \textit{hot-spot cards} de cada \textit{hot-spot}, conforme apresentado por \cite{Fayad1999}.

\section{PathPlanner}

{\large \textbf{Nome do hot-spot:}} Algoritmo de Path Plannig

{\large \textbf{Graus de flexibilidade especificados:}}

\textbf{adaptação sem reinício:} sim

\textbf{adaptação pelo usuário final:} não

{\large \textbf{Descrição geral}}

Definição da trajetória em um mapa já conhecido a partir de um algoritmo estabelecido. Recebe o mapa com os obstáculos e espaços livres, o algoritmo a ser executado e os pontos iniciais e finais e retorna um grafo com os caminhos possíveis

{\large \textbf{Exemplo 1}}

Recebe-se o seguinte mapa, aonde a célula preta significa ocupado, branca significa livre e os pontos iniciais e finais marcados.

\begin{figure}[H]
	\centering
	\label{figXX}
		\includegraphics[keepaspectratio=true,scale=0.7]{figuras/mapahotspot1.PNG}
	\caption{mapa exemplo do \textit{hot-spot card} PathPlanner}
\end{figure}

Para o algorítimo Quadtree gerará o grafo:

\begin{figure}[H]
	\centering
	\label{figXX1}
		\includegraphics[keepaspectratio=true,scale=0.7]{figuras/grafohotspotcard1.PNG}
	\caption{resultado do exemplo 1 do \textit{hot-spot card} PathPlanner}
\end{figure}

{\large \textbf{Exemplo 2}}

Recebe-se o seguinte mapa, aonde a célula preta significa ocupado, branca significa livre e os pontos iniciais e finais marcados.

\begin{figure}[H]
	\centering
	\label{figXX2}
		\includegraphics[keepaspectratio=true,scale=0.7]{figuras/mapahotspot2.PNG}
	\caption{mapa exemplo 2 do \textit{hot-spot card} PathPlanner}
\end{figure}

Para o algorítimo Grafo de Visibilidade gerará o grafo:

\begin{figure}[H]
	\centering
	\label{figXX3}
		\includegraphics[keepaspectratio=true,scale=0.7]{figuras/grafohotspotcard2.PNG}
	\caption{resultado do exemplo 2 do \textit{hot-spot card} PathPlanner}
\end{figure}

\section{BestPath}

{\large \textbf{Nome do hot-spot:}} Algoritmo de Menor caminho

{\large \textbf{Graus de flexibilidade especificados:}}

\textbf{adaptação sem reinício:} sim

\textbf{adaptação pelo usuário final:} não

{\large \textbf{Descrição geral}}

Definição do menor caminho entre um nó e outro em um grafo, retornando a lista de nós com menor custo possível. O grafo e os nós de início e fim são recebidos via construtor e é retornado uma lista de nós.

{\large \textbf{Exemplo 1}}

Recebe-se o grafo abaixo, aonde o inicial é “6,1” e o final é “6,5”.

\begin{figure}[H]
	\centering
	\label{figXX}
		\includegraphics[keepaspectratio=true,scale=0.7]{figuras/grafohotspotcard1.PNG}
	\caption{grafo exemplo do \textit{hot-spot card} BestPath}
\end{figure}

Para o algorítimo Djikstra será gerado o grafo: 6,1 -> 7,2 -> 7,3 -> 6,4 -> 6,5.

{\large \textbf{Exemplo 2}}

Recebe-se o grafo abaixo, aonde o inicial é “6,1” e o final é “6,5”.

\begin{figure}[H]
	\centering
	\label{figXX3}
		\includegraphics[keepaspectratio=true,scale=0.7]{figuras/grafohotspotcard2.PNG}
	\caption{grafo do exemplo 2 do \textit{hot-spot card} BestPath}
\end{figure}

Para o algorítimo A* será gerado o grafo: 6,1 -> 7,1 -> 7,4 -> 6,6.

\section{Map}

{\large \textbf{Nome do hot-spot:}} Modelagem do mapa

{\large \textbf{Graus de flexibilidade especificados:}}

\textbf{adaptação sem reinício:} sim

\textbf{adaptação pelo usuário final:} não

{\large \textbf{Descrição geral}}

Transformação de um modelo de mapa qualquer em uma matriz de inteiros.

{\large \textbf{Exemplo 1}}

Recebe-se a seguinte matriz booleana como mapa inicial:

\begin{table}[H]
\centering
\caption{Tabela de entrada do exemplo 1 do \textit{hot-spot card} Map}
\begin{tabular}{|c|c|c|c|c|c|c|c|}
False & False & False & False & False & False & True & True \\
False & False & False & False & False & False & False & True \\
False & False & False & False & False & False & False & False \\
False & False & False & False & False & False & True & False \\
False & False & False & False & False & True & True & False \\
False & True & True & True & False & False & False & False \\
False & True & True & True & False & False & False & False \\
False & False & False & False & False & False & False & False
\end{tabular}
\end{table}

A matriz de inteiros resultantes será

\begin{table}[H]
\centering
\caption{Tabela de saída do exemplo 1 do \textit{hot-spot card} Map}
\begin{tabular}{|c|c|c|c|c|c|c|c|}
0 & 0 & 0 & 0 & 0 & 0 & 1 & 1 \\
0 & 0 & 0 & 0 & 0 & 0 & 0 & 1 \\
0 & 0 & 0 & 0 & 0 & 0 & 0 & 0 \\
0 & 0 & 0 & 0 & 0 & 0 & 1 & 0 \\
0 & 0 & 0 & 0 & 0 & 1 & 1 & 0 \\
0 & 1 & 1 & 1 & 0 & 0 & 0 & 0 \\
0 & 1 & 1 & 1 & 0 & 0 & 0 & 0 \\
0 & 0 & 0 & 0 & 0 & 0 & 0 & 0
\end{tabular}
\end{table}

aonde 0 é livre e 1 é ocupado.

{\large \textbf{Exemplo 2}}

Para o seguinte mapa topológico recebido

\begin{figure}[H]
	\centering
	\label{figXX3}
		\includegraphics[keepaspectratio=true,scale=0.7]{figuras/grafohotspotcard2.PNG}
	\caption{grafo do exemplo 2 do \textit{hot-spot card} Map}
\end{figure}

A matriz de inteiros resultante será:

\begin{table}[H]
\centering
\caption{Tabela de saída do exemplo 2 do \textit{hot-spot card} Map}
\begin{tabular}{|c|c|c|c|c|c|c|c|}
1 & 1 & 1 & 1 & 1 & 1 & 1 & 1 \\
1 & 1 & 0 & 0 & 0 & 0 & 0 & 0 \\
1 & 0 & 0 & 0 & 0 & 1 & 1 & 0 \\
1 & 0 & 0 & 0 & 0 & 0 & 0 & 0 \\
1 & 0 & 1 & 1 & 1 & 1 & 0 & 0 \\
1 & 0 & 1 & 1 & 1 & 1 & 0 & 0 \\
1 & 0 & 1 & 1 & 1 & 0 & 0 & 1 \\
1 & 0 & 0 & 0 & 0 & 0 & 1 & 1 \\
\end{tabular}
\end{table}

aonde 0 é livre e 1 é ocupado.

\section{EmbeddedCommunicator}

{\large \textbf{Nome do hot-spot:}} Embedded Communication

{\large \textbf{Graus de flexibilidade especificados:}}

\textbf{adaptação sem reinício:} não

\textbf{adaptação pelo usuário final:} não

{\large \textbf{Descrição geral}}

Implementa a comunicação via TCP em hardwares diferentes, utilizando as bibliotecas disponíveis para cada kit.

\chapter{Código fonte}

O código fonte se encontra disponível no seguinte endereço: https://github.com/rodrigorincon/TCC---path-planner-framework/tree/master/codigo/src para o livre uso.

\end{anexosenv}

