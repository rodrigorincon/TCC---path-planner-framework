\begin{resumo}
A robótica móvel procura lidar com máquinas capazes de se locomover de forma autônoma. Para viabilizar o tratamento desse desafio, esse foi dividido em desafios (lê-se objetivos) menores. Um desses objetivos é contemplado nesse trabalho, Trata-se da definição da trajetória do caminho a ser percorrido. Considerando que o ambiente em volta já é conhecido, o robô precisa definir um caminho seguro para chegar aonde deseja, sem se chocar com os obstáculos existentes. Diversos algoritmos com esse intuito foram desenvolvidos, sendo o objetivo deste trabalho prover um \textit{framework} que forneça tais soluções já prontas ao programador. Através de padrões de projetos, o Traveller Framework é projetado para a fácil troca entre os algoritmos, evitando afetar o resto do código. Além da implementação destes algoritmos, é projetada uma arquitetura que permita a portabilidade para o maior número de plataformas, preocupando-se tanto com a reutilização quanto com a legibilidade do código. O Traveller comunica-se com os outros módulos de movimentação do robô por uma interface simples e comum a todos os algoritmos. É importante ressaltar que o módulo de controle do hardware do robô bem como a movimentação dele de fato não fazem parte do escopo deste trabalho. Na verdade, é preocupação desse trabalho a comunicação com o módulo de controle do hardware do robô, fornecendo a esse módulo uma série de coordenadas para onde o robô deve ir. Este trabalho é \textit{open-source}, desenvolvido em linguagem Java, e testado em robôs \textit{differential steering} do kit educacional da LEGO Mindstorm NXT.

 \vspace{\onelineskip}
    
 \noindent
 \textbf{Palavras-chaves}: robótica móvel. definição de trajetória. desenvolvimento de framework. kit educacional da LEGO Mindstorm. Grafo de Visibilidade. Voronoi. Quadtree. Wavefront.
\end{resumo}
